%\documentclass[10pt]{beamer}% 11pt is default, but can try 10pt
\documentclass[10pt,handout]{beamer}

\usefonttheme[onlymath]{serif}
\usepackage[utf8]{inputenc}
\usepackage[T1]{fontenc}
\usepackage{lmodern} % important!

\usepackage{hyperref} % probably already included by beamer package
\hypersetup{pdfauthor={Stephen Becker},
    pdftitle={},
    colorlinks=true,
%    citecolor=MidnightBlue, % requires \usepackage[usenames,dvipsnames,svgnames]{xcolor}
%    urlcolor=Bittersweet,
}


% === Common packages we need sometimes ===
\usepackage{amssymb,amsfonts}
\usepackage{amsmath}
%\usepackage{amsthm} % unnecessary given beamer's theorem class
%\usepackage{color}
%\usepackage{rotating}
%\usepackage{multirow}
%\usepackage{fancybox}
%\usepackage[usenames,dvipsnames,svgnames]{xcolor} % use like \textcolor{red}{..} or {\color{red} ...}
%\usepackage[usenames]{color} % This has ``option clash'' ???
\usepackage{graphicx} % clashes with color, xcolor packages
%\usepackage{booktabs}
%\usepackage{url}
\usepackage{enumitem} % use like \begin{itemize}[noitemsep,nosep,itemindent=-2ex,label=$\circ$]
\usepackage{microtype}	% better font kerning. Not necessary.

%\usepackage[ruled,vlined]{algorithm2e} \usepackage{algorithmic}
%\usepackage{algorithm,algpseudocode} % algorithmicx package -- my preferred

    % create an overlay:
    %\pause

% === Adjust style of slides ===
\mode<presentation>{
    % This means ``hidden'' slides are slightly visible
    % Default is ``invisible'', but you can do
    % transparent=85  (85% is the default)
    % or ``dynamic'', meaning way uncovered text is lighter
%    \setbeamercovered{dynamic}
    \setbeamercovered{transparent=5}
%    \setbeamercovered{transparent}
}
%Themes with side outline: Berkeley, Goettingen, Hannover, Marburg, PaloAlto, 
%\usetheme{Goettingen}
%\usetheme{Hannover}
%\usetheme{Szeged}
\usetheme{Boadilla}
%Themes with top outline: Antibes, Copenhagen, Dresden, Frankfurt, Ilmenau, JuanLesPins, Luebeck, Malmoe, Montpellier, Szeged, Warsaw
% Colors: dolphin, dove (fewer colors), lily, default
% 95% of Boadilla is these lines:
%\usecolortheme{lily}
%\usecolortheme{rose}
%\useinnertheme[shadow]{rounded}
%\usecolortheme{dolphin} % OK
% %\useoutertheme{infolines}
% %\setbeamertemplate{footline}[default]
%\useoutertheme{infolines} % defines colors...
%\setbeamertemplate{headline}[default]
%\setbeamertemplate{footline}[frame number] % this works
% %\setbeamertemplate{footline}[page number]{}  % removes bottom part completely, leaves page numbers

\setbeamertemplate{navigation symbols}{} % remove navigation symbols

%  the rounded inner theme uses a weird ball for itemize which I don't like:
\makeatletter   % for ``@'' character...
%\setbeamertemplate{items}[ball]  % or circle...
\setbeamertemplate{items}[circle]  % or circle...
\makeatother

%\AtBeginSection[]
%\AtBeginSubsection[]
%{\begin{frame}
%        \frametitle{Outline}
%        \small
%        %\tableofcontents[currentsection,hidesubsections]
%        %        \tableofcontents[currentsection,%]%,% % normal one
%        %        currentsubsection]
%        %hideothersubsections,
%        %sectionstyle=show/shaded,
%        %%sectionstyle=show/hide,
%        %subsectionstyle=show/shaded,
%        %]
%        \tableofcontents[currentsection]
%        %\tableofcontents[currentsubsection]
%        \normalsize
%        \addtocounter{framenumber}{-1}
%\end{frame}}



% === Macros ===
\usepackage{xspace}
\newcommand{\thebook}{Shalev-Shwartz and Ben-David\xspace}

\newcommand{\R}{\mathbb{R}}
\newcommand{\E}{\mathbb{E}}
\DeclareMathOperator*{\EE}{\mathbb{E}} % this way allows it to have nice subscript
\renewcommand{\H}{\mathcal{H}}
\renewcommand{\phi}{\varphi}
\newcommand{\bx}{\mathbf{x}}  % b for bold
\newcommand{\x}{\bx}
\newcommand{\bX}{\mathbf{X}}
\newcommand{\X}{\bX}
\newcommand{\by}{\mathbf{y}} 
\newcommand{\y}{\by}
\newcommand{\bz}{\mathbf{z}} 
\newcommand{\z}{\bz}
\newcommand{\ba}{\mathbf{a}}
\newcommand{\bv}{\mathbf{v}}
\newcommand{\bu}{\mathbf{u}}
\newcommand{\bw}{\mathbf{w}}
\newcommand{\w}{\bw}
%\newcommand{\bsigma}{\mathbf{\sigma}}  % b for bold.  This doesn't work
\newcommand{\bsigma}{{\boldsymbol{\sigma}}}  % b for bold
\newcommand{\cX}{\mathcal{X}} % c for caligraphic
\newcommand{\cY}{\mathcal{Y}}
\newcommand{\cD}{\mathcal{D}}
\newcommand{\cO}{\mathcal{O}} % for Order
\newcommand{\order}{\cO} % already defined
\newcommand{\cN}{\mathcal{N}} % for Normal
\newcommand{\loss}{\ell} 	% loss function
\newcommand{\risk}{L} 		% risk, generic
\newcommand{\riskD}{\risk_{\cD}} 	% true risk
\newcommand{\riskEmp}{\widehat{\risk}}
\newcommand{\riskS}{\riskEmp_S}
\newcommand{\Rad}{\mathfrak{R}} % expected Rademacher complexity
\newcommand{\RadEmp}{\widehat{\Rad}} % empirical Rademacher complexity
\DeclareMathOperator{\VC}{VCdim}
\DeclareMathOperator*{\argmin}{argmin}
\newcommand{\defeq}{\stackrel{\text{\tiny def}}{=}} 
\newcommand{\simiid}{\stackrel{\text{\tiny iid}}{\sim}} 
\newcommand{\<}{\langle}  %  useful!!!
\renewcommand{\>}{\rangle}
\newcommand{\iprod}[2]{\left\langle #1 , #2 \right\rangle}
\newcommand{\algo}{\texttt{A}}


% === Title and such ===

\title[Ch 12: inequalities]{Ch 12: Convexity\\APPM 7400 Theory of Machine Learning\\Spring 2020}

\author{Stephen Becker}
\institute[CU]{University of Colorado Boulder}

\date[APPM 7400 Theory of ML]{March 16 2020}

% for the PDF meta data
%\subject{...}
\begin{document}


% === Slide ====================================================================
\begin{frame}
    \titlepage
\end{frame}


% === Section===================================================================
\section{Definitions}


% === Slide ====================================================================-
\begin{frame}{Smoothness and Strong Convexity}
%    \frametitle{Ingredient 1: Sketching}
%    \setbeamercovered{invisible}


The definition of ``\textbf{smoothness}'' in some books (or ``strong smoothness'') of $f$ means  Lipschitz continuity of $\nabla f$  (with constant $L$):
\begin{equation} \label{eq:LL}
    \forall x, y\quad \| \nabla f(x) - \nabla f(y) \| \le L \|x-y\|
\end{equation}
\medskip 

The definition of $f$ being $\mu>0$ \textbf{strongly convex} means that the function $x \mapsto f(x) - \frac{\mu}{2}\|x\|^2$ is convex\footnote{
    See Thm. 5.17 and Remark 5.18 in \cite{BeckBook} --- this is actually only true if $\|\cdot\|$ is the induced norm from the inner product. However, most other properties hold for a general norm.
}. 

\medskip
In the slides below, if $L$ or $\mu$ appears, then we are assuming the gradient is Lipschitz with constant $L$ or $f$ is strongly convex with constant $\mu$, respectively. Most references to Nesterov's book are to his first edition~\cite{Nesterov_2004}, not the recent 2018  edition~\cite{Nesterov_2018}.

\end{frame}

% === Section===================================================================
\section{Inequalities}

% === Slide ====================================================================-
\begin{frame}{Under- and over-approximations}
These two inequalities are very helpful; see, e.g., Thm 2.1.5 and Thm 2.1.10 from \cite{Nesterov_2004}.
\begin{align}
    f(y) &\le f(x) + \iprod{\nabla f(x)}{ y - x } + \frac{L}{2}\|x-y\|^2 \label{eq:L} \\
    f(y) &\ge f(x) + \iprod{\nabla f(x)}{ y - x } + \frac{\mu}{2}\|x-y\|^2 \label{eq:mu}
\end{align}
If we drop convexity but keep Lipschitz continuity of the gradient, then the first equation is still true, but the second equation is not true with $\mu=0$, but it is true with $\mu = -L$.  This is often written as
$\left| f(y) - ( f(x) + \iprod{\nabla f(x)}{ y - x } )\right| \le \frac{L}{2}\|x-y\|^2$.

\bigskip
Related, \cite[Thm.\ 2.1.5, Eq.\ 2.1.10]{Nesterov_2018} gives 
\[
f(y) \ge f(x) + \<\nabla f(x), y-x\> + \frac{1}{2L}\|\nabla f(x)  - \nabla f(y) \|^2 
\]
\end{frame}





% === Slide ====================================================================-
\begin{frame}{Inequalities}
Some nice inequalities can be summarized by: 
{\footnotesize 
\begin{equation} \label{eq:big}
    \left.\begin{aligned}
        L^{-1} \| \nabla f(x) - \nabla f(y) \|^2  \quad\text{\textcolor{red}{(a)}} \\
        \mu \|x-y\|^2 
        \quad\text{\textcolor{red}{(b)}} \\
%        \frac{\mu L}{\mu + L} \|x-y\|^2  + \frac{1}{\mu+L}\| \nabla f(x) - \nabla f(y) \|^2
%        \quad\text{\small\textcolor{red}{(c)}}
    \end{aligned} \right\rbrace
    \le \< \nabla f(x) - \nabla f(y), x-y \> 
    \le 
    \left\lbrace\begin{aligned}
        \text{\textcolor{red}{(d)}} \quad& L \|x-y\|^2  \\
        \text{\textcolor{red}{(e)}} \quad& \mu^{-1}  \| \nabla f(x) - \nabla f(y) \|^2
    \end{aligned} \right.
\end{equation}
}

The inequality {\small\textcolor{red}{(a)}} %left-most $\le$ above in the line for $L$ 
really follows from the co-coercivity of gradients; this result is actually surprisingly strong, since it makes implicit use of the Baillon-Haddad theorem. The result {\small\textcolor{red}{(e)}} for $\mu$ also requires $f$ be continuously differentiable. 

\medskip
We can actually get a tighter lower bound if we assume \emph{both} strong convexity and Lipschitz continuity of the gradient; see \cite[Thm. 2.1.12]{Nesterov_2004} for a derivation. That result is:
\[
\frac{\mu L}{\mu + L} \|x-y\|^2  + \frac{1}{\mu+L}\| \nabla f(x) - \nabla f(y) \|^2
\le \< \nabla f(x) - \nabla f(y), x-y \> 
\]


%The {\small\textcolor{red}{(c)}} inequality assumes both strong convexity and Lipschitz continuity of the gradient; see \cite[Thm. 2.1.12]{Nesterov_2004} for a derivation.


\end{frame}


% === Slide ====================================================================-
\begin{frame}{Sub-optimality bounds}

 For unconstrained smooth optimization, if $x^\star$ is a minimizer, then $\nabla f(x^\star) = 0$. Note there are 3 equivalent definitions of optimality: $x$ is optimal if
\begin{equation}
    \|x - x^\star\|=0, \quad
    f(x) - f^\star = 0, \quad
    \|\nabla f(x)\| = 0
\end{equation}
%and this would be ``iff'' if we assume the optimal solution is unique.

\medskip If we change all the zeros above to $\epsilon > 0$, are these conditions equivalent? 
On the next slides, we'll investigate this.

\medskip To start with, here's a first result: note that since the gradient is in the subdifferential, combined with H\"older's inequality, then (\cite[\S2.2.2]{Nesterov_2018})
\begin{equation}
    f(x)-f^\star \le \|\nabla f(x)\|_p \|x-x^\star\|_{p'}\quad (\forall p,p' \text{ s.t. } 1/p + 1/p' = 1 )
\end{equation}
which doesn't require Lipshitz continuity or strong convexity. This can be useful if it is known $x$ lies in a bounded set, since then $\|x-x^\star\|$ can be bounded.

\end{frame}

% === Slide ====================================================================-
\begin{frame}{Sub-optimality bounds: assuming strong smoothness}
If $f$ has a $L$-Lipschitz continuous derivative, we can bound
\begin{align}
    \|\nabla f(x) \| = \| \nabla f(x) - \nabla f(x^\star)\| &\le L \|x-x^\star\| \quad \text{by \eqref{eq:LL}}\\
    f(x) - f^\star &\le \frac{L}{2}\|x-x^\star\|^2
    \quad\text{by \eqref{eq:L}} \\
    \|\nabla f(x) \|^2 &\le 2L \left( f(x) - f^\star\right) \quad\text{by Eq. (9.14) in \cite{BoydVandenbergheBook}} \label{eq:33}
\end{align}

Note further that $f$ must be twice-continuously differentiable to apply 
\eqref{eq:33} is proved in \cite{BoydVandenbergheBook} assuming $f$ is twice-differentiable, but without assuming twice differentiability it can be proved using \cite[Thm.\ 2.1.5, Eq.\ 2.1.10]{Nesterov_2018}.
    
\end{frame}

% === Slide ====================================================================-
\begin{frame}{Sub-optimality bounds: assuming strong convexity}
Assuming $f$ is $\mu>0$ strong convexity, we can bound in the other direction:
\begin{align}
    \|x-x^\star\|^2 &\le \frac{1}{\mu^2}\|\nabla f(x)\|^2 
    \quad \text{by \eqref{eq:big}  {\small\textcolor{red}{(b)}} and  {\small\textcolor{red}{(e)}} }
    \\
    \|x-x^\star\|^2 &\le \frac{2}{\mu}\left( f(x) - f^\star\right)
    \quad \text{by \eqref{eq:mu}, with $x=x^\star$, $y=x$} \label{eq:10} \\
    f(x) - f^\star &\le \frac{1}{2\mu}\|\nabla f(x)\|^2 \quad\text{by Eq. (9.9) in \cite{BoydVandenbergheBook}. This is PL}
    \label{eq:44}
\end{align}
Note: at least Eq.~\eqref{eq:10} holds for any norm~\cite[Thm. 5.25]{BeckBook}.

 Note: \eqref{eq:44} is the Polyak-Lojasiewicz (PL) inequality, see \href{https://arxiv.org/abs/1608.04636}{Karimi, Nutini, Schmidt} for details. 
 
 

\end{frame}


% === Slide ====================================================================-
\begin{frame}{References}
\begin{thebibliography}{AZQRY16}
    
    \bibitem[BC11]{bauschke2011convex}
    H. H. Bauschke and P.~L. Combettes, \href{https://link.springer.com/book/10.1007/978-1-4419-9467-7}{\emph{Convex analysis and monotone
            operator theory in {H}ilbert spaces}, 1st edition}, {S}pringer, 2011.
    
    \bibitem[BC17]{bauschke2017convex}
    H. H. Bauschke and P.~L. Combettes, \href{https://link.springer.com/book/10.1007/978-3-319-48311-5}{\emph{Convex analysis and monotone
            operator theory in {H}ilbert spaces}, 2nd edition}, {S}pringer, 2017.
    
    \bibitem[Bec17]{BeckBook}
    A. Beck, \href{https://epubs.siam.org/doi/book/10.1137/1.9781611974997?mobileUi=0}{First-Order Methods in Optimization}, SIAM, 2017.
    
    \bibitem[BV04]{BoydVandenbergheBook}
    S.~Boyd and L.~Vandenberghe.
    \newblock \href{http://www.stanford.edu/~boyd/cvxbook/}{{\em Convex Optimization}}.
    \newblock Cambridge University Press, 2004.
    
    \bibitem[Nes04]{Nesterov_2004}
    Yu.~Nesterov.
    \newblock \href{https://link.springer.com/book/10.1007/978-1-4419-8853-9}{{\em Introductory Lectures on Convex Optimization: A Basic Course}},
    volume~87 of {\em Applied Optimization}.
    \newblock Kluwer, Boston, 2004.
    
    \bibitem[Nes18]{Nesterov_2018}
    Yu. Nesterov.
    \newblock \href{https://link.springer.com/book/10.1007/978-3-319-91578-4}{{\em Lectures on Convex Optimization}}.
    \newblock Springer International Publishing, 2018.
    
\end{thebibliography}
    
\end{frame}

\end{document}